ProStruct: A modern C++ protein structure analysis library with extendible interfaces

Abstract


Introduction

The programming language of choice for many scientists is usually dynamically typed, as it 
is easier to learn and to apply. This comes at a cost of execution speed and the lack
of type checking can lead to errors. Several packages have been written for protein
structure analysis for python and other interpreted language and they leverage 
the speed given by code written in C/C++ or other compiled languages. However, these
libraries are written around the target interpreted language, which makes the design
less efficient and more difficult to extend in the backend. 
This paper introduces a new protein structure analysis tool written in C++17 which takes
advantage of the newest developments of the C++ programming language in order to make it easier
to extend and make more efficient. In order to make it accessible to most scientists, Prostruct
also has interfaces written in SWIG (Simplified Wrapper and Interface Generator). This wrapper
provides a set a C/C++ functions that connect the backend in C++ with the interpreted languages.

Methods

A protein structure can be broken down to several elements: the atoms, their bonds, the amino acids 
(or residues), the chain and the overall structure. These are all represented by C++ classes
which are exposed to the interfaces. 

At its core ProStruct has a so-called kernel engine that applies a function along a protein sequence.
A kernel is represented by a C++ lambda or function, but the former has a better performance. The advantage
of this is that any function can be easily represented by a user and the engine takes care of 
the execution and is optimised by the compiler.

TALK ABOUT DIRECTOR CLASS

TALK ABOUT DSL

Applications

